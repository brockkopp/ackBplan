\section{Executive Summary}

The executive summary is a standalone document that is often provided first to a 
potential investor.

The goal of the executive summary is to get potential investors sufficiently interested 
in your business opportunity to dig further into your business plan or take a meeting 
and hear your presentation. Investors see many opportunities so your executive 
summary must stand out in the first few sentences. 

Think of the executive summary as your business plan in two pages. We have 
adapted the following outline from Guy Kawasaki�s Reality Check. The headings cover 
the components that should be included in the executive summary.

Example: Executive summary outline
- Problem: What pressing and important problem are you solving or 
what opportunity are you addressing?

- Solution: How are you solving this problem or tapping this 
opportunity?

- Business model: Who are your customers and how will you make 
money?

- Product and technology: What is your product or technology? What 
is the underlying magic? What makes your company special?

- Marketing and sales strategy: What is your go-to-market strategy?

- External environment and competition: Who do you compete 
with? What can you do that they cannot? What can they do that you 
cannot?

- Financial analysis and projections: What are your financial 
projections for the next three to five years? What are your key 
assumptions and metrics to achieve these projections?

- Team: Who is on your team? Why are they special?

- Implementation roadmap: Where are you now? What are the 
major and immediate milestones?

Since the executive summary is a standalone document, it should be more 
attractively designed (e.g., colours, graphics, layout) than the rest of the business 
plan document.
