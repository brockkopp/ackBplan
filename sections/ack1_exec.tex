\section{Executive Summary}

The construction industry is notoriously slow at adopting new technologies which could revolutionize even their most standard processes and result in substantial and immediate cost savings. One of these processes is that of tracking and approving technical drawings. As a project is completed, drawings finalized through the collaboration of many different stakeholders. These include the architect, engineers, the client, sub-contractors and the general contractors themselves. The general contractor is responsible for managing the approval of these documents and assuring that all document are not only approved, but approved quickly such that the sub-contractors are not waiting idly for approvals. Efficiency is especially crucial in new design-build projects where construction is started before the building is finalized. In such projects, entire sites can be on hold waiting for a critical aspect to be finalized and approved. Any such delays can result in massive costs to be born by either the client or contractor.

{\it Acknowledgements} will revolutionize this process by offering a full document approval and tracking system. By migrating document transmission to a single online service, the process of sharing drawings will be greatly simplified. Robust version control will assure that all parties are working from the same documents, and that no one is ever working from an out of date drawing. Not only will document revisions be managed, but the entire process of having stakeholders approve revisions of documents will be managed from a single system.

At the beginning of a project, the general contractor will create a new {\it Acknowledgements} project. During this process they will select the different people responsible for aspects of the project, and the standard flow of documents during approval processes. {\it Acknowledgements} will be accessible not only by the general contractor, but the architecture, engineering and other related firms will all be able to login to a central system. Once the project is initialized, drawings will be added to the system. Drawings will then be automatically ``delivered'' to different stakeholders though {\it Acknowledgements} rather than by standard mail or E-mail. Since drawings will always be transmitted through {\it Acknowledgements}, the state of each document will be known at all times. 

General contracting companies will be the initial target clients for {\it Acknowledgements}. As the company coordinating a construction project, they have the most to gain by implementing {\it Acknowledgements}. If a general contractor is using {\it Acknowledgements}, all associated contractors will also be required to use the interface. This will force the distribution of {\it Acknowledgements} throughout the industry as multiple general contractors will use the same architecture or engineering firms. While {\it Acknowledgements} will be initially designed specifically for the construction industry, many other possible application exist in any industry where technical drawings are prevalent and undergo constant revisions (i.e. manufacturing). 

As a web application, the distribution of {\it Acknowledgements} is greatly simplified. The application will be managed from a central hosting facility, with client and technical support located in close proximity. Being centrally hosted has the additional benefit of relieving the client of most technical support issues. Any issue will be quickly resolved by {\it Acknowledgements} specialists.

There are currently no large document approval management systems commercially available. There are many revision control or document sharing applications, but none of them integrate an approval system capable of meeting the strict legal liability requirements of the construction industry. {\it Acknowledgements} will integrate the approval process while simplifying management of the entire process. While no commercial solutions exist, there are some instances of comparable systems developed by general contractors for use within their respective companies. Sources within the industry report that these systems are outdated and cumbersome to use. This clearly demonstrates the need for a system such as {\it Acknowledgements}, except one developed professionally by a firm which is able to perform continuous improvement as web technologies evolve.

Due to the widely varying size of general contractors, the pricing scheme for {\it Acknowledgements} will be very flexible. The benefit of using {\it Acknowledgements} to a company grows proportional to the number of currently active projects as well as the specific set of features required by the company. As such the pricing of {\it Acknowledgements} will also increase proportional to the number of active projects for a company and requested features. While more complicated than standardized pricing schemes, this model will be attractive to companies no matter the size of the contracts. Companies will also be able to mitigate perceived risk by scaling the use of {\it Acknowledgements} within their company as they realize the benefits of using the system.

{\it Acknowledgements} will be led by and executive team of engineering graduates with extensive experience in both the construction industry as well as in the development of web applications for a variety of other industries. The team will leverage this extensive experience to bring best practices from other industries to further revolutionize {\it Acknowledgements}. Extensive technical experience is complemented by practical managerial and supplier relations experience. 

{\it Acknowledgements} is currently in its early development phases. Initial application designs have been competed and will be implemented in the next four months. With the application implemented, rigorous process validation as well as user testing will begin in September 2013. User testing will be facilitated by the hiring of a user interface design specialist. Once the product is ready for market, clients will be sought throughout the industry. Client development will initially be focused in Ontario, and will expand throughout Canada over the following year.