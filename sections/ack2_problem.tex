\section{The Problem}

\subsection{Market Overview}
The construction industry is a notoriously slow moving and with few exceptions a laggard in adopting new technologies. While selling a new product is initially hard in this environment, there is huge potential once early adopters have been found. The Canadian Construction Association estimates the value of all approved Canadian building permits to be \$5.85 Billion as of January 2013.

Despite being such a large industry, major general contractors still fail to take full advantage of full electronic document management and approval systems. Many of these general contractors either fail to make use of electronic document tracking, or rely on systems developed within the company. While systems developed within a company are highly customizable, these companies simply do not have the capabilities in-house to develop and maintain systems which take advantage of constantly evolving web technology.

\subsection{Market Challenges}

The construction industry is very stable industry which has exhibited significant growth over the past decade. Statistics Canada estimates that the value of building permits issued across Canada have grown from \$ 24.5 Million in 1995 to \$ 61 Million in 2009. However as the industry grows, assuring internal efficiency does not suffer is critical to remaining competitive. It is because of this growth that a system such as {\it Acknowledgements} is critical for large corporations to maintain document approval efficiency across hundreds of concurrent projects.

While large corporations dominate the market, Defence Construction Canada estimated that 90\% of construction firms still employ fewer than 20 people. These firms manage few projects at any one time, but organizational efficiency is even more critical due to limit resources. With a slow economy hurting industries across Canada, everyone is looking to streamline their internal processes and come out on top.

\subsection{Market Opportunity}
Positioning itself to help a growing industry that is trying to improve efficiency, {\it Acknowledgements} will revolutionize electronic document tracking and approval processes. By migrating processes from traditional paper based systems and bulky in-house developed systems, firms will be able to realize significant improvements in document approval times. These benefits of these efficiencies will greatly outweigh the continuing cost of subscribing to {\it Acknowledgements}.

Document is not only a problem for a general contractor, but for all of their dependent partners as well. While a general contractor must coordinate a project, they are constantly managing documents as they move between:

\begin{itemize}
	\item Clients
	\item Architects
	\item Civil Engineers
	\item Sub-contractors
\end{itemize}

Coordinating the movement of these documents and their current status in the approval process is a very complicated and time consuming process. By centralizing the state of each document as its is passed between stakeholders, {\it Acknowledgements} will greatly simplify this process, without compromising integrity. By performing this coordination via a secure web-based application, each stakeholder will be able to access a unified interface and thus quite literally be ``on the same page''.