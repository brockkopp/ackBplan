\section{Product and Technology}
Acknowledgements will rely heavily on new web application technologies such as HTML5 and new scripting languages. These tools will allow Acknowledgements to be developed quickly and minimize the cost of additional features. While significant effort was required to implement functionality such as ``drag-and-drop'' a few years ago, this can not be implemented in minutes with new tools available. These new web technologies will however merely be tool to accomplish the user-friendly interface Acknowledgements will be known for.

Acknowledgements will use dynamic scripting to generate a customized home interface for each user. These customized interfaces will display only the projects and functionality relevant to that user's job. While additional functionality (given proper permissions) will be easily accessible, the selected core functionality will be all that is readily displayed. By focusing on the user's experience, Acknowledgements will be able to gain popularity not only impress individuals involved in the procurement process, but gain popularity with the employees who will use Acknowledgements every day. Client loyalty will be critical in not only retaining existing customers, but also soliciting referrals to new potential clients.

Since all document processes will be managed through Acknowledgements, many analytics are possible which are infeasible using paper based systems. In todays online world, data is currency. Companies using Acknowledgements will be able to leverage collected process data to further streamline their procedures. Acknowledgements will be able to produce analytics identifying trends and outliers at the user, project and even company level. This data can help managers identify bottle-necks in the process and determine the root cause. While these reports can be run manually, Acknowledgements will be monitoring project progress, and will be able to notify a supervisor if a specific task is taking longer than normal for the given comapany.

These unique features will be based on proprietary algorithms which are capable of identifying issues early in their development, so that they can be fixed before the client is severely affected.