\section{Risk Analysis}

{\bf Acknowledgements} is positioned to enter a niche market with currently sparse competition. While this may be true now, with success competition is soon to quickly enter the market. As such 3 critical success factors have been identified to assure that Acknowledgements evolves into a market leader for technical document tracking and approval.

The industry reputation of {\bf Acknowledgements} will be based on a clean and user-friendly interface which allows a user to easily utilize the powerful and secure management and tracking system that corporations will rely on. As such it will be paramount that all releases of the software are thoroughly tested not only for process integrity, but also ease of use. Many applications are known to have started with a simple and clean interface, only to become cluttered and confusing as functionality is added. User testing and closed loop feedback will be important to not only gaining, but sustaining a reputation as the most productive product on the market.

One of the largest weaknesses of web-applications is server failure. Any down-time will have a significant negative impact on the reputation of {\bf Acknowledgements}. Risk of application down-time will be mitigated by using a distributed hosting service, where {\bf Acknowledgements} is hosted by one company, but in various geographic locations. This will result in a much more robust hosting solution that will be resistant to technological failures or ``acts of god''. Another cause of down-time is an application update which causes a critical failure. In order to mitigate this potential issue, all updates will be performed with a full live backup of the previous software release ready to be reinstated. As such, if there is an issue with the update, the old version will be reinstated within minutes of the error being detected. Update cycles will also make use of weekends to assure that updates are performed during periods of minimal application use.

Due to the nature of liability in the construction industry, {\bf Acknowledgements'} approval system will only truly be tested in the event of a catastrophe. It will be during these times that it is critical that {\bf Acknowledgements} is able to produce a full report of exactly which documents were approved by who and in what sequence to assure that liability can be assigned clearly and accurately. Failing to provide an accurate report of the approval process will significantly undermine the reputation of {\bf Acknowledgements} and could immediately put the company's reputation in disrepute. Integrity testing will be constantly undertaken with projects selected at random. This will assure that any issues are discovered before they come to light.

% \begin{center} \begin{tabular}{| p{1in} | p{1.75in} | p{3in} | }
\begin{center} \begin{tabular}{| p{2in} | p{4in} | }
    \hline
    % {\bf Type of Risk} & {\bf Risk} & {\bf Mitigating Strategy} \\ \hline
    {\bf Risk} & {\bf Mitigating Strategy} \\ \hline
	
    % {Product} & 
	{Product does not have required functionality} & 
	{This possibility has been deemed very unlikely due to the foreseen simplicity of implementation. If the product is not ready according to development schedule, more time will be taken rather than rushing an incomplete product to market.} \\ \hline
	
    % {Market Adoption} & 
	{\raggedright Client not interested in trying {\bf Acknowledgements}} & 
	{Contacts within existing companies will be leveraged to assure the product meets client requirements before release.} \\ \hline
	
    % {Market Size Risk} & 
	{The existing market is smaller than anticipated} & 
	{Thorough market research has proven an existing market for {\bf Acknowledgements}. Acknowledgements also has a potential market in the manufacturing industry where technical drawings also require constant revisions and approval.} \\ \hline
	
    % {Financing Risk} & 
	{Insufficient funds to develop or sustain {\bf Acknowledgements}} & 
	{Development plans minimize development cost to avoid this. Alternative financing will be sought from financial institutions should this occur once clients have been established.} \\ \hline

    % {Execution Risk} & 
	{{\bf Acknowledgements} is the first corporation for this management team.} & 
	{The Acknowledgement executive team have worked together for years on various initiatives. They will supplement their knowledge with help from industry advisors.} \\ \hline
\end{tabular} \end{center}