\section{Business Model}
A business model describes an organization�s value to its customers. It illustrates the capabilities and resources required to create, market and deliver this value and to generate profitable, sustainable revenue streams. The business model is important because it describes how you will make money with your venture.

Most of the information you require to describe your business model will have been developed in Market Strategy Development Workbook 2: Critical Value Factors and Market Strategy Development Workbook 3: Strategic Marketing Approach.

This section should include the following details:
- how your business model works
- the value proposition
- the target market
- key partnerships
- pricing and positioning
- the distribution model

Refer to the information you have documented in Market Strategy
Development Workbook 2: Critical Value Factors.

\subsection{content}
Acknowledgement will differentiate itself from existing solutions by developing a product known for enchancing productivity and being creating a ``frustration-free'' work environment. Acknowledgements will rely heavily on client referrals in the early stages of its development. It is through these referrals that Acknowledgements will leverage its acceptance at a one company to acquire futher clients.

\subsection{Partnerships}
Our future involves partnerships with A Small Orange web hosting company, LibreOffice, a software development company and Mint.com, a secure application development company.

\begin{enumerate}
  \item {\bf Mint.com} will be our first partnership. Security is an important part of our clients' workflow. Mint.com is experienced in creating secure applications, and they will be able to help us ensure that our application is secure.
  \item {\bf A Small Orange} will be our second partner, and will be pursued once our product is done development. All our services will be hosted on their servers. They offer great prices and great services and will be a valuable part of our business.
  \item {\bf LibreOffice} will be a long-term partner. Once our product is established with a starting client base, we will expand the offerings of our product to include a fully integrate office suite in our online software.
\end{enumerate}


\subsection{Distribution}
The distribution of Acknowledgements is greatly simplified as a web application. The service will be accessible by multiple clients from a single central location. If a client would prefer to host the service on-site for security or other client-specific reasons. Assuming a client is using the centrally hosted option, the additional cost per client will be negligigle during early stages of deployment. Common web hosting solutions are capable of serving hundreds of clients without the need to upgrade. As Acknowledgements grows, distribution costs will increase marginally, but be significantly less than aditional client revenue.

While hosting the application centrally will greatly reduce distribution costs, it also increases risk since there will be a single failure point. This risk will be mitigated by utilizing a large hosting company such as Amazon, which allows the deployment of Acknowledgements to be managed centrally, but have redundant hosting centers in case of server failure. 

\subsection{Pricing}
While Acknowledgements will become a critical system for the client, they will be initially unaware of their need. As such Acknowledgement's pricing scheme will emphasize initial low entry costs, with costs growing proportionally with the client's dependance on the application. Client revenue is proportional to the size of the company's current contracts. As such the cost of Acknowledgements will increase on a per-project basis. This scheme will be logical to a client since all drawings and other document will all be project specific.

Acknowledgements pricing will therefore be divided into three categories:
\begin{itemize}
\item \textbf{Trial} - All potential clients will be eligible for a trial of the software, free of charge. During this trial the client will have full application functionality, for the period of the project. The trial will allow a potential client to try the software, with minimal risk.
\item \textbf{Ongoing Fees} - Upon completion of the trial, the company will be charged a yearly fee based on the current number of projects being managed through Acknowledgements. This pricing scheme will allow company's to slowly adopt Acknowledgements, without large upfront costs. As user acceptance grows, the application will become critical to the client's business model.
\item \textbf{Extended Functionality} - The development of additional features will possible at any point after the client has become an ongoing Acknowledgements client. Additional features will be avaliable for an upfront development cost in addition to a small ongoing yearly fee.
\end{itemize}

