\section{The Solution}
Acknowledgements is a web-based document tracking platform, facilitating secure sharing and approval of technical documents. In a world where liability is paramount, having a clear record of a document�s evolution is critical for assuring the safety and integrity of the end product. A simple web interface will allow for systematic approval of a document. The system will allow for collaboration within an organization as well as with external partners, any of whom can sign off on a document throughout its life-cycle.

Acknowledgments differentiates itself from existing solutions by emphasizing simplicity and ease of use. Most solutions on the market are made in-house by companies and are rarely complete, effective or user-friendly. Significant improvements have been made in web application technologies and Acknowledgements will leverage these technologies to simplify the user's experience. User acceptance has been identified as a significant challenge to be overcome by Acknowledgements, and as such a new user must realize the advantages of using Acknowledgements quickly before them become discouraged.

\subsection{Development}
Acknowledgement will primarily be a document management application with the ability to automate document distribution between many parties. This core functionality will be complemented by enhanced security which will allow users to �approve� documents, while maintaining all industry legal accountability requirements.

The initial release of the product will only contain this core functionality. However each company will have new and unique requirements for Acknowledgements given their specific situation. These additional features (or ``modules'') will be developed for the company for an additional charge. Once each additional module has been developed, it will also be available to other clients, again for an additional charge. As such the capabilities of Acknowledgements will grow with industry requirements.

\subsection{Competition}
Acknowledgements will be entering an industry largely lacking competition. As such, it will be important to enter the market with a fully developed product that will capture market share quickly. Once a company has adopted Acknowledgements, it will be difficult to switch an application which has become entrenched in the company's work processes, due to employee retraining. As such it will be difficult for an alternative product to enter the market after Acknowledgement and compete with what will have become and industry standard.

Since Acknowledgements will not only be used by general contractors, but also the many third parties working under the general contractor (architects, engineers and sub-contractors), switching applications will only be made more difficult once these third parties adjust to Acknowledgements.